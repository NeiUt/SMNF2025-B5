% Options for packages loaded elsewhere
\PassOptionsToPackage{unicode}{hyperref}
\PassOptionsToPackage{hyphens}{url}
\PassOptionsToPackage{dvipsnames,svgnames,x11names}{xcolor}
%
\documentclass[
  12pt,
  a4paper,
]{article}

\usepackage{amsmath,amssymb}
\usepackage{iftex}
\ifPDFTeX
  \usepackage[T1]{fontenc}
  \usepackage[utf8]{inputenc}
  \usepackage{textcomp} % provide euro and other symbols
\else % if luatex or xetex
  \usepackage{unicode-math}
  \defaultfontfeatures{Scale=MatchLowercase}
  \defaultfontfeatures[\rmfamily]{Ligatures=TeX,Scale=1}
\fi
\usepackage{lmodern}
\ifPDFTeX\else  
    % xetex/luatex font selection
\fi
% Use upquote if available, for straight quotes in verbatim environments
\IfFileExists{upquote.sty}{\usepackage{upquote}}{}
\IfFileExists{microtype.sty}{% use microtype if available
  \usepackage[]{microtype}
  \UseMicrotypeSet[protrusion]{basicmath} % disable protrusion for tt fonts
}{}
\makeatletter
\@ifundefined{KOMAClassName}{% if non-KOMA class
  \IfFileExists{parskip.sty}{%
    \usepackage{parskip}
  }{% else
    \setlength{\parindent}{0pt}
    \setlength{\parskip}{6pt plus 2pt minus 1pt}}
}{% if KOMA class
  \KOMAoptions{parskip=half}}
\makeatother
\usepackage{xcolor}
\usepackage[margin=2.5cm]{geometry}
\setlength{\emergencystretch}{3em} % prevent overfull lines
\setcounter{secnumdepth}{5}
% Make \paragraph and \subparagraph free-standing
\makeatletter
\ifx\paragraph\undefined\else
  \let\oldparagraph\paragraph
  \renewcommand{\paragraph}{
    \@ifstar
      \xxxParagraphStar
      \xxxParagraphNoStar
  }
  \newcommand{\xxxParagraphStar}[1]{\oldparagraph*{#1}\mbox{}}
  \newcommand{\xxxParagraphNoStar}[1]{\oldparagraph{#1}\mbox{}}
\fi
\ifx\subparagraph\undefined\else
  \let\oldsubparagraph\subparagraph
  \renewcommand{\subparagraph}{
    \@ifstar
      \xxxSubParagraphStar
      \xxxSubParagraphNoStar
  }
  \newcommand{\xxxSubParagraphStar}[1]{\oldsubparagraph*{#1}\mbox{}}
  \newcommand{\xxxSubParagraphNoStar}[1]{\oldsubparagraph{#1}\mbox{}}
\fi
\makeatother

\usepackage{color}
\usepackage{fancyvrb}
\newcommand{\VerbBar}{|}
\newcommand{\VERB}{\Verb[commandchars=\\\{\}]}
\DefineVerbatimEnvironment{Highlighting}{Verbatim}{commandchars=\\\{\}}
% Add ',fontsize=\small' for more characters per line
\usepackage{framed}
\definecolor{shadecolor}{RGB}{241,243,245}
\newenvironment{Shaded}{\begin{snugshade}}{\end{snugshade}}
\newcommand{\AlertTok}[1]{\textcolor[rgb]{0.68,0.00,0.00}{#1}}
\newcommand{\AnnotationTok}[1]{\textcolor[rgb]{0.37,0.37,0.37}{#1}}
\newcommand{\AttributeTok}[1]{\textcolor[rgb]{0.40,0.45,0.13}{#1}}
\newcommand{\BaseNTok}[1]{\textcolor[rgb]{0.68,0.00,0.00}{#1}}
\newcommand{\BuiltInTok}[1]{\textcolor[rgb]{0.00,0.23,0.31}{#1}}
\newcommand{\CharTok}[1]{\textcolor[rgb]{0.13,0.47,0.30}{#1}}
\newcommand{\CommentTok}[1]{\textcolor[rgb]{0.37,0.37,0.37}{#1}}
\newcommand{\CommentVarTok}[1]{\textcolor[rgb]{0.37,0.37,0.37}{\textit{#1}}}
\newcommand{\ConstantTok}[1]{\textcolor[rgb]{0.56,0.35,0.01}{#1}}
\newcommand{\ControlFlowTok}[1]{\textcolor[rgb]{0.00,0.23,0.31}{\textbf{#1}}}
\newcommand{\DataTypeTok}[1]{\textcolor[rgb]{0.68,0.00,0.00}{#1}}
\newcommand{\DecValTok}[1]{\textcolor[rgb]{0.68,0.00,0.00}{#1}}
\newcommand{\DocumentationTok}[1]{\textcolor[rgb]{0.37,0.37,0.37}{\textit{#1}}}
\newcommand{\ErrorTok}[1]{\textcolor[rgb]{0.68,0.00,0.00}{#1}}
\newcommand{\ExtensionTok}[1]{\textcolor[rgb]{0.00,0.23,0.31}{#1}}
\newcommand{\FloatTok}[1]{\textcolor[rgb]{0.68,0.00,0.00}{#1}}
\newcommand{\FunctionTok}[1]{\textcolor[rgb]{0.28,0.35,0.67}{#1}}
\newcommand{\ImportTok}[1]{\textcolor[rgb]{0.00,0.46,0.62}{#1}}
\newcommand{\InformationTok}[1]{\textcolor[rgb]{0.37,0.37,0.37}{#1}}
\newcommand{\KeywordTok}[1]{\textcolor[rgb]{0.00,0.23,0.31}{\textbf{#1}}}
\newcommand{\NormalTok}[1]{\textcolor[rgb]{0.00,0.23,0.31}{#1}}
\newcommand{\OperatorTok}[1]{\textcolor[rgb]{0.37,0.37,0.37}{#1}}
\newcommand{\OtherTok}[1]{\textcolor[rgb]{0.00,0.23,0.31}{#1}}
\newcommand{\PreprocessorTok}[1]{\textcolor[rgb]{0.68,0.00,0.00}{#1}}
\newcommand{\RegionMarkerTok}[1]{\textcolor[rgb]{0.00,0.23,0.31}{#1}}
\newcommand{\SpecialCharTok}[1]{\textcolor[rgb]{0.37,0.37,0.37}{#1}}
\newcommand{\SpecialStringTok}[1]{\textcolor[rgb]{0.13,0.47,0.30}{#1}}
\newcommand{\StringTok}[1]{\textcolor[rgb]{0.13,0.47,0.30}{#1}}
\newcommand{\VariableTok}[1]{\textcolor[rgb]{0.07,0.07,0.07}{#1}}
\newcommand{\VerbatimStringTok}[1]{\textcolor[rgb]{0.13,0.47,0.30}{#1}}
\newcommand{\WarningTok}[1]{\textcolor[rgb]{0.37,0.37,0.37}{\textit{#1}}}

\providecommand{\tightlist}{%
  \setlength{\itemsep}{0pt}\setlength{\parskip}{0pt}}\usepackage{longtable,booktabs,array}
\usepackage{calc} % for calculating minipage widths
% Correct order of tables after \paragraph or \subparagraph
\usepackage{etoolbox}
\makeatletter
\patchcmd\longtable{\par}{\if@noskipsec\mbox{}\fi\par}{}{}
\makeatother
% Allow footnotes in longtable head/foot
\IfFileExists{footnotehyper.sty}{\usepackage{footnotehyper}}{\usepackage{footnote}}
\makesavenoteenv{longtable}
\usepackage{graphicx}
\makeatletter
\def\maxwidth{\ifdim\Gin@nat@width>\linewidth\linewidth\else\Gin@nat@width\fi}
\def\maxheight{\ifdim\Gin@nat@height>\textheight\textheight\else\Gin@nat@height\fi}
\makeatother
% Scale images if necessary, so that they will not overflow the page
% margins by default, and it is still possible to overwrite the defaults
% using explicit options in \includegraphics[width, height, ...]{}
\setkeys{Gin}{width=\maxwidth,height=\maxheight,keepaspectratio}
% Set default figure placement to htbp
\makeatletter
\def\fps@figure{htbp}
\makeatother
% definitions for citeproc citations
\NewDocumentCommand\citeproctext{}{}
\NewDocumentCommand\citeproc{mm}{%
  \begingroup\def\citeproctext{#2}\cite{#1}\endgroup}
\makeatletter
 % allow citations to break across lines
 \let\@cite@ofmt\@firstofone
 % avoid brackets around text for \cite:
 \def\@biblabel#1{}
 \def\@cite#1#2{{#1\if@tempswa , #2\fi}}
\makeatother
\newlength{\cslhangindent}
\setlength{\cslhangindent}{1.5em}
\newlength{\csllabelwidth}
\setlength{\csllabelwidth}{3em}
\newenvironment{CSLReferences}[2] % #1 hanging-indent, #2 entry-spacing
 {\begin{list}{}{%
  \setlength{\itemindent}{0pt}
  \setlength{\leftmargin}{0pt}
  \setlength{\parsep}{0pt}
  % turn on hanging indent if param 1 is 1
  \ifodd #1
   \setlength{\leftmargin}{\cslhangindent}
   \setlength{\itemindent}{-1\cslhangindent}
  \fi
  % set entry spacing
  \setlength{\itemsep}{#2\baselineskip}}}
 {\end{list}}
\usepackage{calc}
\newcommand{\CSLBlock}[1]{\hfill\break\parbox[t]{\linewidth}{\strut\ignorespaces#1\strut}}
\newcommand{\CSLLeftMargin}[1]{\parbox[t]{\csllabelwidth}{\strut#1\strut}}
\newcommand{\CSLRightInline}[1]{\parbox[t]{\linewidth - \csllabelwidth}{\strut#1\strut}}
\newcommand{\CSLIndent}[1]{\hspace{\cslhangindent}#1}

\makeatletter
\@ifpackageloaded{caption}{}{\usepackage{caption}}
\AtBeginDocument{%
\ifdefined\contentsname
  \renewcommand*\contentsname{Table of contents}
\else
  \newcommand\contentsname{Table of contents}
\fi
\ifdefined\listfigurename
  \renewcommand*\listfigurename{List of Figures}
\else
  \newcommand\listfigurename{List of Figures}
\fi
\ifdefined\listtablename
  \renewcommand*\listtablename{List of Tables}
\else
  \newcommand\listtablename{List of Tables}
\fi
\ifdefined\figurename
  \renewcommand*\figurename{Figure}
\else
  \newcommand\figurename{Figure}
\fi
\ifdefined\tablename
  \renewcommand*\tablename{Table}
\else
  \newcommand\tablename{Table}
\fi
}
\@ifpackageloaded{float}{}{\usepackage{float}}
\floatstyle{ruled}
\@ifundefined{c@chapter}{\newfloat{codelisting}{h}{lop}}{\newfloat{codelisting}{h}{lop}[chapter]}
\floatname{codelisting}{Listing}
\newcommand*\listoflistings{\listof{codelisting}{List of Listings}}
\makeatother
\makeatletter
\makeatother
\makeatletter
\@ifpackageloaded{caption}{}{\usepackage{caption}}
\@ifpackageloaded{subcaption}{}{\usepackage{subcaption}}
\makeatother

\ifLuaTeX
  \usepackage{selnolig}  % disable illegal ligatures
\fi
\usepackage{bookmark}

\IfFileExists{xurl.sty}{\usepackage{xurl}}{} % add URL line breaks if available
\urlstyle{same} % disable monospaced font for URLs
\hypersetup{
  pdftitle={SMNF2025-B5},
  pdfauthor={Mattis Becker; Justus H; Tommy P; Aliaksandra F},
  colorlinks=true,
  linkcolor={blue},
  filecolor={Maroon},
  citecolor={Blue},
  urlcolor={Blue},
  pdfcreator={LaTeX via pandoc}}


\title{SMNF2025-B5}
\author{Mattis Becker \and Justus H \and Tommy P \and Aliaksandra F}
\date{2025-05-13}

\begin{document}
\maketitle

\renewcommand*\contentsname{Table of contents}
{
\hypersetup{linkcolor=}
\setcounter{tocdepth}{3}
\tableofcontents
}

\#\#Link des Repositories

https://github.com/NeiUt/SMNF2025-B5/tree/main

\#\#Aktueller Hash: -965b7bb0ad9f8d6a27db51c1a0c41527b04bd42d

\section{Code of Conduct}\label{code-of-conduct}

\begin{itemize}
\item
  Umgang mit Feedback, unterschiedlichen Perspektiven und
  Meinungsverschiedenheiten:

  \begin{itemize}
  \tightlist
  \item
    Feedback wird ehrlich aufgenommen und bei zukünftigen Arbeiten
    angewendet
  \end{itemize}
\item
  Faire Aufteilung der Arbeitslast:

  \begin{itemize}
  \tightlist
  \item
    Die zu erledigenden Aufgaben werden gerecht und in Absprache
    aufgeteilt.
  \end{itemize}
\item
  Verhalten in Bezug auf vereinbarte und verpflichtende Termine:

  \begin{itemize}
  \tightlist
  \item
    Termine für gemeinsames projektorientiertes Arbeiten, werden per
    Online-Messanger vereinbart und die Gruppenmitglieder finden sich
    dann in einem Videoanruf ein oder es wird sich in Präsenz getroffen.
  \end{itemize}
\item
  Einhaltung der wissenschaftlichen Integrität:

  \begin{itemize}
  \tightlist
  \item
    Es werden nur verlässliche Quellen verwendet, die auch korrekt
    zitiert werden und geschriebenes wird objektiv verfasst.
  \end{itemize}
\item
  Verpflichtung zum Schutz von Daten und zur Wahrung ihrer
  Vertraulichkeit:

  \begin{itemize}
  \tightlist
  \item
    Daten werden vertaulich behandelt und nicht an Dritte weitergegeben.
  \end{itemize}
\item
  Nutzung und Kennzeichnung von AI Tools (z.B. ChatGPT):

  \begin{itemize}
  \tightlist
  \item
    Nutzung von AI tools wird gekennzeichnet und nicht direkt
    übernommen.
  \end{itemize}
\end{itemize}

\begin{center}\rule{0.5\linewidth}{0.5pt}\end{center}

\section{Einleitung}\label{einleitung}

\emph{(Einleitung eingefügen)}

\section{Literaturübersicht}\label{literaturuxfcbersicht}

Fake News verbreiten sich oft durch manipulierte Bilder, Deepfakes und
bearbeiteten Ton. Diese Inhalte können schnell die öffentliche Meinung
beeinflussen, Vertrauen zerstören und Verschwörungstheorien verstärken.
Deshalb ist Medienkompetenz sehr wichtig.

Ein automatisiertes System wurde entwickelt, um Fake News in Tamil zu
erkennen. Dafür wurde ein Datensatz mit Überschriften, Texten und
Bildern erstellt. Die Nachrichten wurden in „true``, „false`` und
„fake`` eingeteilt.

Zur Verarbeitung kamen Transformer-Modelle für Text und Bild zum
Einsatz. Ein siamesisches Modell prüfte, ob Text und Bild
zusammenpassen. LLMs erstellten automatisch Bildbeschreibungen zur
besseren Analyse.

Das System erreichte eine gute Erkennungsrate (F1-Score: 0,8736). Mit
erklärbarer KI wurden die Entscheidungen des Modells verständlich
gemacht. (Lekshmi Ammal and Madasamy 2025)

Die Arbeit ``Requirements engineering for artificial intelligence
systems: A systematic mapping study'' befasst sich insbesondere mit der
Fragestellung, wie Anforderungsstellung an moderne KI-Systeme erfplgt
und welche verfügbaren Frameworks, Methoden, Werkzeuge und Techniken
dazu verwendet werden und welche Limitierungen und Herausforderungen der
Anforderungsstellung sich dabei vorfinden lassen.

Basierend auf 43 referenzierten Studien wurden die darin verwendeten
Methoden, Modelle, Werkzeuge und Techniken der Anforderungsstellung
analysiert.

Basierend auf der Analyse wird in der Arbeit ersichtlich, dass
betrachtete KI-Systeme mangelnde Intergration aktueller
Anforderungsmodelle, Werkzeuge und Methoden aufweisen. Nach der Analyse
der 43 Studien ergab sich, dass es noch viele Herausforderungen in der
Anforderungsstellung an KI-Systeme gibt, wobei besondere Schwerpunkte
die Erklärbarkeit generierter Inhalte, ethische Fragen, Fragen der
Datenanforderungen und Mangel an Kommunikation zwischen
Softwareentwicklern und Data Scientists. (Ahmad et al. 2023)

Die Arbeit der Autoren Farnaz Jahanbakhsh, Yannis Katsis, Dakuo Wang,
Lucian Popa und Michael Müller hat sich mit der Fragestellung
beschäftigt, wie menschliche Bewertungen und KI-Vorhersagen zusammen
genutzt werden können, um Fehlinformationen in Online-Beiträgen zu
erkennen. Dabei geht es darum, die Zusammenarbeit zwischen Nutzern und
einer personalisierten künstlichen Intelligenz (KI) zu untersuchen, um
die Genauigkeit bei der Bewertung von Beiträgen, besonders auf sozialen
Medien, zu verbessern und mögliche Probleme zu Analysieren. Das Ziel ist
herauszufinden, ob eine solche personalisierte KI, Menschen dabei helfen
kann, die Vertrauenswürdigkeit von Online-Inhalten besser einzuschätzen.
Es wurde untersucht, wie Nutzer mit der vorhersage der KI interagieren,
ob sie davon profitieren und oder ob Probleme auftreten könnten. Ein
Vorteil der personalisierten KI ist, dass sie Nutzer als helfer dienen
kann. Sie kann dem Nutzer helfen potentielle Falschinformationen zu
erkennen, bevor diese weitergeleitet werden. Jedoch gibt auch
Herausforderungen, denn die KI könnte fehlerhafte Vorhersagen machen,
die die Bewertung des Nutzers beeinflussen. So könnten falsche
Informationen als glaubwürdig erkannt werden. Des Weiteren gibt es die
Befürchtung, dass der Einsatz so einer KI, insbesondere durch
Plattformbetreiber, die Meinungsfreiheit einschränken könnte. Das wird
zum einen von Wissenschaftlern als auch zum anderen von Nutzern kritisch
gesehen. Die KI analysiert, wie ein Nutzer in der Vergangenheit Inhalte
eingeschätzt hat, und erstellt daraus eine Person bezogene vorhersage.
Dieses Modell sagt dann voraus, wie der jeweilige Nutzer wahrscheinlich
aktuelle Inhalte bewerten würde. Etwa ob ein Tweet wahr oder eine
Fehlinformation ist. Die Nutzerstudie zeigt, dass eine personalisierte
KI die Nutzer durch ihre Vorhersagen Beeinflusst. Dieser Effekt
verschwand jedoch, wenn die Nutzer der Bewertung nachgingen durch eine
Begründung ihrer Entscheidung

(Jahanbakhsh et al. 2023)

\section{Methode}\label{methode}

\emph{(Methode einfügen)}

\section{Ergebnisse}\label{ergebnisse}

\emph{(Ergebnisse einfügen)}

\section{Diskussion}\label{diskussion}

\subsection{\texorpdfstring{\emph{(Diskussion
einfügen)}}{(Diskussion einfügen)}}\label{diskussion-einfuxfcgen}

\subsection{Quarto}\label{quarto}

Quarto enables you to weave together content and executable code into a
finished document. To learn more about Quarto see
\url{https://quarto.org}.

\subsection{Running Code}\label{running-code}

When you click the \textbf{Render} button a document will be generated
that includes both content and the output of embedded code. You can
embed code like this:

\begin{Shaded}
\begin{Highlighting}[]
\DecValTok{1} \SpecialCharTok{+} \DecValTok{1}
\end{Highlighting}
\end{Shaded}

\begin{verbatim}
[1] 2
\end{verbatim}

You can add options to executable code like this

\begin{verbatim}
[1] 4
\end{verbatim}

The \texttt{echo:\ false} option disables the printing of code (only
output is displayed).

``Tommy05lol''

``DefoNotAlex''

``Matt1s1234''

``NeiUt''

\phantomsection\label{refs}
\begin{CSLReferences}{1}{0}
\bibitem[\citeproctext]{ref-Quelle2}
Ahmad, Khlood, Mohamed Abdelrazek, Chetan Arora, Muneera Bano, and John
Grundy. 2023. {``Requirements Engineering for Artificial Intelligence
Systems: A Systematic Mapping Study.''} \emph{Information and Software
Technology} 158: 107176.
https://doi.org/\url{https://doi.org/10.1016/j.infsof.2023.107176}.

\bibitem[\citeproctext]{ref-jahanbakhsh2023}
Jahanbakhsh, Farnaz, Yannis Katsis, Dakuo Wang, Lucian Popa, and Michael
Muller. 2023. {``Exploring the Use of Personalized AI for Identifying
Misinformation on Social Media.''} In \emph{Proceedings of the 2023 CHI
Conference on Human Factors in Computing Systems}. CHI '23. New York,
NY, USA: Association for Computing Machinery.
\url{https://doi.org/10.1145/3544548.3581219}.

\bibitem[\citeproctext]{ref-LekshmiAmmal2025}
Lekshmi Ammal, Hariharan Ramakrishna Iyer, and Anand Kumar Madasamy.
2025. {``A Reasoning Based Explainable Multimodal Fake News Detection
for Low Resource Language Using Large Language Models and
Transformers.''} \emph{Journal of Big Data} 12 (1): 46.
\url{https://doi.org/10.1186/s40537-025-01093-x}.

\end{CSLReferences}




\end{document}
